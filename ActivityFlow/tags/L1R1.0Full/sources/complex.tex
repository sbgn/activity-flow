\subsection{Glyph: \glyph{Complex activity}}
\label{sec:af:complex}

A \glyph{complex activity} node represents the activity from a biochemical entity composed of other biochemical entities, whether macromolecules, simple chemicals, multimers, or other complexes. For example, a heterotetramer of a voltage-gated potassium channel has a channel pore formed by four different subunits. On the other hand, if the activity is known to come from a particular component of the complex, the macromolecule activity node should be used.

\begin{glyphDescription}

\glyphSboTerm SBO:

\glyphContainer A \glyph{complex activity} is represented by a rectangle with cut-corners (an octagonal box with sides of two different lengths).  Individual subunits or component of the complex should not be represented.

\glyphLabel The identification of a named \glyph{complex activity} is carried by an unbordered box containing a string of characters.  The characters may be distributed on several lines to improve readability, although this is not mandatory.  The label box has to be attached to the midway between the border of the complex's container box and the border of the components' container boxes.

\glyphAux  A \glyph{complex activity} can also carry one or several \glyph{units of information} (see \sect{af:unitInfo}).

\end{glyphDescription}

\begin{figure}[H]
  \centering
%  \includegraphics[scale = 0.3]{images/complex}
  \caption{An example \AF glyph for \glyph{complex activity}.}
  \label{fig:af:complex}
\end{figure}
