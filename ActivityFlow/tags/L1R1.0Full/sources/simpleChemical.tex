\subsection{Glyph: \glyph{Simple chemical activity}}
\label{sec:af:simpleChemicalActivity}

A simple chemical activity in SBGN Actifity Flow is defined as the activity from a chemical compound that is not formed by the covalent linking of pseudo-identical residue, as opposite of a macromolecule activity (\sect{af:macromolecule}). Examples of simple chemicals are an atom, a monoatomic ion, a salt, a radical, a solid metal, a crystal, etc.

\begin{glyphDescription}

\glyphSboTerm SBO:

\glyphContainer A \glyph{simple chemical activity} is represented by a circular container, as depicted in \fig{af:simpleChemicalActivity}.

\glyphLabel The identification of the \glyph{simple chemical activity} is carried by an unbordered box containing a string of characters.  The characters may be distributed on several lines to improve readability, although this is not mandatory.  The label box has to be attached to the center of the circular container.  The label is permitted to spill outside the container.

\glyphAux A \glyph{simple chemical activity} may be decorated with one or more \glyph{units of information} (\sect{af:unitInfo}). 

\end{glyphDescription}

\begin{figure}[H]
  \centering
  \includegraphics[scale = 0.3]{images/simpleChemical}
  \caption{The \AF glyph for \glyph{simple chemical activity}.}
  \label{fig:af:simpleChemicalActivity}
\end{figure}
