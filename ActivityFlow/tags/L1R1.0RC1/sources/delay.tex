%%%%%%%%%%%%%%%%%%%%%%%%%%%%%%%%%%%%%%%%%%%%%%%%%%%%%%%%%%%%%%%%%%%%%%
%%                     delay
%%%%%%%%%%%%%%%%%%%%%%%%%%%%%%%%%%%%%%%%%%%%%%%%%%%%%%%%%%%%%%%%%%%%%%

\subsubsection{Glyph: \glyph{delay}}\label{sec:delay}

The glyph \glyph{delay} is used to denote that the \glyph{activity node} linked as input does not produce the influence immediately.

\begin{glyphDescription}
 \glyphSboTerm SBO:NEW ! delay.
 \glyphOrigin More than one AN (\sect{af:ANs}) or logical operator (section~\ref{sec:af:logic}).
 \glyphTarget  Modulation arc (\sect{af:arcs}).
 \glyphContainer \glyph{Delay} is represented by a circle
 \glyphLabel \glyph{Delay} is identified by the greek letter ``$\tau$`` (``TAU'') placed in an unbordered box attached to the center of the container.
 \glyphAux \glyph{Delay} does not carry any auxiliary items.
\end{glyphDescription}

\begin{figure}[H]
  \centering
  \includegraphics[scale = 0.5]{images/delay}
  \caption{The \ER glyph for \glyph{delay}.}
  \label{fig:delay}
\end{figure}
\normalcolor
