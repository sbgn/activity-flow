\chapter{Activity Flow language grammar}
\label{chp:af:grammar}

\section{Overview}
In this chapter, we describe how the glyphs of SBGN \AF can be combined to make a valid SBGN \AF map. To do this, we must at the very least define what glyphs can be connected to each other. This is called syntax. Next, we must define rules over and above connection rules, such as whether duplicate symbols are permitted. In addition, we must define what the notation ``means'' -- how does it represent a biological pathway? This is semantics, and it is essential if a reader is to understand a SBGN map without external help, and a writer is to create one that reflects his understanding of a biological system.

In this section we start off by describing the concepts of the \AF{} notation. Next a detailed description of the syntax is provided
followed by a description of the semantic rules of the notation.

\section{Concepts}

The SBGN \AF{} is more than a collection of symbols. It is a visual language that uses specific abstractions to describe the biological activities that make up a model, a signaling pathway or a metabolic network. This abstraction is the semantics of SBGN, and to describe it requires more than a definition of the symbols and syntax of the language. 

The \AF{} in SBGN describes biological activities involving biological entities. A \emph{biological activity} can influence, or be influenced by, other \glyph{biological activities}, and such relationships are represented in \AF by lines with arrows and other decorations. So the essence of \AF is to show the flow of activities from one entity to another or within the same entity. The underlying mechanisms of how the influence occurs may not be known and is not captured in the diagram. If the mechanism is known, the details should be described in annotation or captured in other SBGN languages, such \PD and/or \ER.

\section{Syntax}

The syntax of the SBGN \AFl is defined in the form of an incidence matrix. An incidence matrix has arcs as rows and nodes as columns. Each element of the matrix represents the role of an arc in connection to a node. Input (I) means that the arc can begin at that node. Output (O) indicates that the arc can end at that node. Numbers in parenthesis represent the maximum number of arcs of a particular type to have this specific connection role with the node. Empty cells means the arc is not able to connect to the node.

\subsection{Activity Nodes connectivity definition}
\begin{tabular}{||c|c|c|c|c|c|c|c|c|c||}
\hline
\hline
\raisebox{20pt}{$Arc \backslash Node $}   &\vglyph{biological activity}   & \vglyph{pertubation}  & \vglyph{phenotype}    & \vglyph{tag}  & \vglyph{submap}  & \vglyph{and} & \vglyph{or} & \vglyph{not} & \vglyph{delay}  \\ \hline
\glyph{positive influence}              & I \& O                        & I                     & O                     &               &   & I & I & I & I \\ \hline
\glyph{negative influence}              & I \& O                        & I                     & O                     &               &   & I & I & I & I \\ \hline
\glyph{unknown influence}               & I \& O                        & I                     & O                     &               &   & I & I & I & I \\ \hline
\glyph{necessary stimulation}           & I \& O                        & I                     & O                     &               &   & I & I & I & I \\ \hline
\glyph{logic arc}                       & I                             &                       &                       &               &   & O & O & O & O \\ \hline
\glyph{equivalence arc}                 & I                             &                       &                       & O             & O & & & &  \\ 
\hline 
\hline
\end{tabular}

\subsection{Containment definition}
There are two node types allowing containment of AF in SBGN: \glyph{compartment} and \glyph{submap}. The next table describe relationship between AF elements of SBGN and these two containers. Plus sign means that the element is able to be contained within a container. An empty cell means containment is not allowed. \\

\begin{tabular}{||c|c|c||}
\hline
\hline
$ elements \backslash Containers$     & \glyph{compartment}   & \glyph{submap}    \\ \hline
\glyph{biological activity}     &         +             &          +        \\ \hline
\glyph{pertubation}             &         +             &          +        \\ \hline
\glyph{phenotype}               &         +             &          +        \\ \hline
\glyph{tag}                     &         +             &          +        \\ \hline
\glyph{compartment}             &         -             &          +        \\ \hline
\glyph{submap}                  &         +             &          +        \\ \hline
\glyph{positive influence}      &         +             &          +        \\ \hline
\glyph{negative influence}      &         +             &          +        \\ \hline
\glyph{unknown influence}       &         +             &          +        \\ \hline
\glyph{logic arc}               &         +             &          +        \\ \hline
\glyph{equivalence arc}         &         +             &          +        \\ \hline
\glyph{and}                     &         +             &          +        \\ \hline
\glyph{or}                      &         +             &          +        \\ \hline
\glyph{not}                     &         +             &          +        \\ \hline
\hline
\end{tabular}


\subsection{Syntactic rules}

There are additional syntactic rules that must be applied in addition to those defined above.

\subsubsection{AFs}

 \begin{enumerate}
    \item Each \glyph{biological activity} has at most one \glyph{unit of information}.
    \item \glyph{Pertubation} is never target of a modulation arc.
    \item \glyph{Phenotype} is never source of a modulation arc.
 \end{enumerate}





\section{Semantic rules}

%\subsection{Namespaces}

%The notation has a concept of a namespace within which activities with the same identifying attributes are regarded as identical. The SBGN namespaces are shown in table \ref{tab:namespacedefs}.

%\begin{center}
%\tablecaption{Namespace scope definitions.}
%\label{tab:namespacedefs}
%\begin{small}
%\tablefirsthead{\hline
%  Namespace Scope & Activity influenced & Notes\\\hline}
%\tablehead{\hline
%\multicolumn{3}{|l|}{\small\sl continued from previous page}\\
%\hline\hline
%  Namespace scope & Activity influenced & Notes\\\hline\hline}
%\tabletail{\hline
%\multicolumn{3}{|r|}{\small\sl continued on next page}\\
%\hline}
%\tablelasttail{\hline}
%\begin{supertabular}{|l|p{5cm}|p{5cm}|}\hline

%MapDiagram & CompartmentNode, SubMapDiagram, EquivalenceNode & \\\hline

%CompartmentShape & ActivityNode & If no \glyph{compartment} is drawn then all ActivityNodes are assumed to belong to an invisible ``default'' compartment.\\\hline
%ActivityType & UnitOfInformation & \\\hline
%\end{supertabular}
%\end{small}
%\end{center}


\subsection{Activity node and Compartment}

Each AN can only appear once in a particular compartment.  An AN can only \emph{belong} to one compartment. However, an AN can be \emph{drawn} over more than one compartment. In such cases the decision on which is the owning compartment is deferred to the drawing tool or the author. 

The layout of compartments in an SBGN diagram does not imply anything about the topology of compartments in the cell. Compartments should be bounded and may overlap. However, adjacency and the nesting of compartments does not imply that these compartments are next to each other physically or that one compartment contains the other.

\subsection{Modulation}

It is implied, but not defined explicitly that an activity has a rate at which the input AN effects the output AN. 

\begin{enumerate}
\item Positive influence is a modulation that's effect is to increase the basal activity.
\item Negative influence is a modulation that's effect is to decrease the basal activity.
\item Unknown influence is a modulation where the effect and basal activity is unknown.
\item At most one necessary stimulation can be assigned to an activity. Two necessary stimulations
  would imply an implicit Boolean AND or OR operator. For clarity only
  one necessary stimulation can be assigned to an activity node and such combinations must be
  explicitly expressed as the Boolean operators.
\end{enumerate}

\subsection{Submaps}

Submaps are a visual device that allow a map to be split into several views. They remain, however, part of the main map and share its
namespace. As a test of validity it should be possible to reintroduced a submap into the main map by eliminating the SubMapNode and merging
the equivalent nodes in both maps.

\subsubsection{Rules for mapping to submaps}

An AN in the main map can be mapped to one in the submap using a TagNode in the submap and SubMapTerminals (see \sect{submap}) in the main map. For a
mapping between map and submap to exists the following must be true:

\begin{enumerate}
\item The identifiers in the TagNode and SubMapTerminals must be identical.
\item The AFNs must be identical.
\end{enumerate}

\subsubsection{Requirement to define a mapping}

If a map and submap both contain the same AN, then a mapping between them must be defined as above.

