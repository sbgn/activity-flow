\chapter{Preface}

\section*{Acknowledgements}

The authors are grateful to all the attendees of the SBGN meetings, as well as to the subscribers of the \mailto{sbgn-discuss@sbgn.org} mailing list.  
% The authors would like to acknowledge especially the help of Frank Bergmann, Sarala Dissanayake, Ralph Gauges, Peter 
% Ghazal, and Lu Li. SM and AS would also like to acknowledge Igor Goryanin whose financial support and encouragement 
% enabled us to commit the necessary time to the development of this specification.
% A more comprehensive list of people involved in SBGN development is available in the
% appendix~\ref{sec:af:acknowledgments}.

The development of SBGN was mainly supported by a grant from the Japanese \emph{New Energy and Industrial Technology Development Organization} (NEDO, \url{http://www.nedo.go.jp/}).  The \emph{Okinawa Institute of Science and Technology} (OIST, \url{http://www.oist.jp/}), the \emph{AIST Computational Biology Research Center} (AIST CBRC, \url{http://www.cbrc.jp/index.eng.html}) the British \emph{Biotechnology and Biological Sciences Research Council} (BBSRC, \url{http://www.bbsrc.ac.uk/}) through a Japan Partnering Award, the European Media Laboratory (EML Research gGmbH, \url{http://www.eml-r.org/}), and the Beckman Institute at the California Institute of Technology (\url{http://bnmc.caltech.edu}) provided additional support for SBGN workshops.  Some help was provided by the \emph{Japan Science and Technology Agency} (JST, \url{http://www.jst.go.jp/}) and the \emph{Genome Network Project} of the Japanese Ministry of Education, Sports, Culture, Science, and Technology (MEXT, \url{http://www.mext.go.jp/}) for the development of the gene regulation network aspect of SBGN, and from the \emph{Engineering and Physical Sciences Research Council} (EPSRC, \url{http://www.epsrc.ac.uk}) during the redaction of the specification.

\section*{Notes on typographical conventions}

The concept represented by a glyph is written using a normal font, while a \glyph{glyph} means the SBGN visual representation of the concept. 