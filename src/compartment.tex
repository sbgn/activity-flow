A compartment is a logical or physical structure where the function or activity is located.  At the moment, an activity can only belong to one compartment. Therefore, the ``same'' biochemical activities located in two different compartments are in fact, two different activities, and should be represented separately.

\begin{glyphDescription}

\glyphSboTerm  SBO:0000289 ! functional compartment

\glyphIncoming None.

\glyphOutgoing Zero or more \glyph{equivalence arcs} (\sect{equivalenceArc}).

\glyphContainer A compartment is represented by a surface enclosed in a continuous border or located between continuous borders. These borders should be noticeably thicker than the borders of the ANs. A compartment can take \textbf{any} geometry. A compartment must always be entirely enclosed.

\glyphLabel The identification of the compartment is carried by an unbordered box containing a string of characters. The characters can be distributed on several lines to improve readability, although this is not mandatory. The label box can be attached anywhere in the container box. Note that the label can spill over from the container box.

\glyphAux A \glyph{compartment} can carry a certain number of \glyph{units of information}, that will add information, for instance, about the physical environment, such as pH, temperature or voltage, see \sect{af:unitInfoComp}.  The centre of the bounding box of a \glyph{unit of information} must lie on the border of the compartment.

\end{glyphDescription}

\begin{figure}[H]
  \centering
  \includegraphics[scale = 1]{images/build/compartment.pdf}
  \caption{The \AF glyph for \glyph{compartment}.}
  \label{fig:af:compartment}
\end{figure}

It is important to note that a compartment never contains another compartment. To allow more aesthetically pleasing and understandable diagrams, compartments are allowed to overlap each other visually, but it must be kept in mind that this does not mean one compartment contains part or entire of the other compartment.  \fig{overlap} shows three semantically equivalent placements of compartments:

\begin{figure}[H]
  \centering
  \includegraphics[scale = 0.4]{images/build/compartment_overlapping_example.pdf}
  \caption{Overlapped compartments are permitted, but the overlap does not imply containment.}
  \label{fig:overlap}
\end{figure}

