%%%%%%%%%%%%%%%%%%%%%%%%%%%%%%%%%%%%%%%%%%%%%%%%%%%%%%%%%%%%%%%%%%%%%%
%%                     delay
%%%%%%%%%%%%%%%%%%%%%%%%%%%%%%%%%%%%%%%%%%%%%%%%%%%%%%%%%%%%%%%%%%%%%%

\subsection{Glyph: \glyph{Delay}}\label{sec:delay}

The \glyph{delay} glyph represents that the input, connected via a logic arc from a \glyph{BA} or the output of another logical operator, does not produce its influence immediately on the target activity.


\begin{glyphDescription}
 \glyphSboTerm SBO:0000225 ! delay.
 \glyphIncoming One \glyph{logic arc} (\sect{af:logicArc}).
 \glyphOutgoing  One \glyph{logic arc} (\sect{af:logicArc}) or one of the influence arcs (\sect{af:arcs}).
 \glyphContainer A \glyph{delay} operator is represented by a circular shape containing the symbol ``$\uptau$`` (letter ``tau'' of the Greek alphabet).
The shape is linked to two ports, that are small arcs attached to the centres of opposite sides of the shape, as shown in \fig{af:delay}.
 \glyphLabel None.
 \glyphAux None.
\end{glyphDescription}

\begin{figure}[H]
  \centering
  \includegraphics[scale = 0.6]{images/build/delay.pdf}
  \caption{The \AF glyph for \glyph{delay}.}
  \label{fig:af:delay}
\end{figure}

\begin{figure}[H]
  \centering
  \includegraphics[scale = 0.8]{images/build/delay_example.pdf}
  \caption{An example of the \glyph{delay} logical operator, where the activity of \glyph{p21} is positively influenced after a delay by the activity of \glyph{p53}, modelling the time-dependent transcriptional activation of p21 by p53.}
  \label{fig:af:ex-or}
\end{figure}
