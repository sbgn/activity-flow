% =============================================================================
% introduction
% =============================================================================

\chapter{Introduction}
\label{sec:intro}

With the rise of systems and synthetic biology, the use of graphical
representations of pathways and networks to describe biological
systems has become pervasive. It was therefore important to use a
consistent notation that would allow people to interpret those maps
easily and quickly, without the need of extensive
legends. Furthermore, distributed investigation of biological systems
in different labs as well as activities like synthetic biology, that
reconstruct biological systems, need to exchange their descriptions
unambiguously, as engineers exchange circuit diagrams.

The goal of the Systems Biology Graphical Notation (SBGN) is to
standardize the graphical/visual representation of biochemical and
cellular processes. SBGN defines comprehensive sets of symbols with
precise semantics, together with detailed syntactic rules defining
their use. It also describes the manner in which such graphical
information should be interpreted. SBGN is made up of three different
and complementary languages \cite{LeNovere:NatBiotechnol:2009}. This document
defines the \emph{\AF} visual language of SBGN. \AFs are one of three
views of a biological process offered by SBGN. It is the product of
many hours of discussion and development by many individuals and
groups.
 
\section{What are the languages?}

The \textbf{Process Description} language permits the description of all the processes taking place in a biological system. The \textbf{Entity Relationship} language permits the description of all the relations involving the entities of a biological system. The \textbf{Activity Flow} language permits the description of the flow of activity in a biological system.

\section{Nomenclature}

The three languages of SBGN should be referred to as:
\begin{itemize}[noitemsep, nolistsep]
\item the Process Description language (the PD language).
\item the Entity Relationship language (the ER language).
\item the Activity Flow language (the AF language).
\end{itemize}

A specific representation of a biological system in one of the SBGN languages should be referred to as:
\begin{itemize}[noitemsep, nolistsep]
\item a Process Description map (a PD map).
\item an Entity Relationship map (an ER map).
\item an Activity Flow map (an AF map).
\end{itemize}

The corpus of all SBGN representations should be referred to as:
\begin{itemize}[noitemsep, nolistsep]
\item Process Descriptions.
\item Entity Relationships.
\item Activity Flows.
\end{itemize}

The capitalisation is important. PD, ER and AF are names of languages. As such they must be capitalised in English. This is not the case of the accompanying noun (language or map).

\section{SBGN levels and versions}
\label{sec:sbgn-levels}

It was unquestionable at the outset of SBGN development that it would be impossible to design a perfect and complete notation right from the beginning. Apart from the prescience this would require (which, sadly, none of the authors possess), it also would likely need a vast language that most newcomers would shun as being too complex. Thus, the SBGN community followed an idea used in the development of other standards, i.e. stratify language development into levels.

A \emph{level} of one of the SBGN languages represents a set of features deemed to fit together cohesively, constituting a usable set of functionality that the user community agrees is sufficient for a reasonable set of tasks and goals. Within \emph{levels}, \emph{versions} represent the evolution of a language, which may involve new glyphs and refined semantics, but no fundamental change of the way maps are to be generated and interpreted. In addition, new versions should be backwards compatible, \ie \AF maps that conform to an earlier version of the \AFl within the same level should still be valid. This does not apply to a new level.

Capabilities and features that cannot be agreed upon and are judged
insufficiently critical to require inclusion in a given level, are
postponed to a higher level or version. In this way, the development
of SBGN languages is envisioned to proceed in stages, with each higher
level adding richness compared to the levels below it.

\section{Developments, discussions, and notifications of updates}
\label{sec:discussions}

The SBGN website (\url{http://sbgn.org/}) is a portal for all things 
related to SBGN. It provides a web forum interface to the SBGN discussion list (\mailto{sbgn-discuss@googlegroups.com}) and information about how anyone may subscribe to it. The easiest and best way to get involved in SBGN discussions is to join the mailing list and participate.

Face-to-face meetings of the SBGN community are announced on the website as well as the mailing list. Although no set schedule currently exists for workshops and other meetings, we envision holding at least one public workshop per year. As with other similar efforts, the workshops are likely to be held as satellite workshops of larger conferences, enabling attendees to use their international travel time and money more efficiently.

Notifications of updates to the SBGN specification are also broadcast on 
the mailing list and announced on the SBGN website.

\section{Note on the typographical conventions and requirement levels}
\label{sec:notes}
The concept represented by a glyph is written using a regular font, while a 
\glyph{glyph} means the SBGN visual representation of the concept. For 
instance, ``a biological activity is encoded by the SBGN AF \glyph{biological activity}.''

Throughout this specification, we use two requirement levels, indicated by the keywords ``must'' and ``should'':
\begin{enumerate}
  \item requirements, i.e., rules which \textbf{must} be fulfilled, and
  \item recommendations, i.e., rules which \textbf{should} be followed if
  possible. 
\end{enumerate}


\section{Structure of this document} \chap{af:glyphs} provides a catalogue of the graphical symbols available for representing entities in \AFs. In \chap{af:grammar} beginning on page~\pageref{chp:af:grammar}, we describe the rules for combining these glyphs into a legal SBGN \AFm, and in \chap{af:layout} beginning on page~\pageref{chp:af:layout}, we describe requirements and guidelines for the way that \AFms are visually organized.
