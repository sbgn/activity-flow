\subsection{Glyph: \glyph{Not}}
\label{sec:af:not}

The output of a \glyph{not} glyph is True if its input is False, and False otherwise.

\begin{glyphDescription}
 \glyphSboTerm SBO:0000238 ! not.
 \glyphIncoming One \glyph{logic arc} (\sect{af:logicArc}).
 \glyphOutgoing  One \glyph{logic arc} (\sect{af:logicArc}) or one of the influence arcs (\sect{af:arcs}).
 \glyphContainer A \glyph{not} operator is represented by a circular shape containing the word ``NOT''.
The shape is linked to two ports, that are small arcs attached to the centres of opposite sides of the shape, as shown in \fig{af:not}.
 \glyphLabel None.
 \glyphAux None.
 \end{glyphDescription}

\begin{figure}[H]
  \centering
  \includegraphics[scale = 1]{images/build/not.pdf}
  \caption{The \AF glyph for \glyph{not}.}
  \label{fig:af:not}
\end{figure}

\begin{figure}[H]
  \centering
  \includegraphics[scale = 0.8]{images/build/not_example.pdf}
  \caption{An example of the \glyph{not} logical operator, where the activity of \glyph{NR1D1} has a negative influence on the activity of \glyph{Cdkn1a} taking place in the regulation of the circadian clock.}
  \label{fig:af:ex-not}
\end{figure}
