\subsection{Glyph: \glyph{Phenotype}}
\label{sec:af:phenotype}

A phenotype is a type of biological process.  In SBGN, \emph{phenotype} is used to show the observable or measurable outcome of the network.  It is usually the end-point(s) of the network, i.e., it cannot be used as the start of an arc.

\begin{glyphDescription}

\glyphSboTerm
SBO:0000358 ! phenotype

\glyphIncoming Zero or more \glyph{influence} arcs (\sect{af:arcs}).

\glyphOutgoing None.

\glyphContainer A \glyph{phenotype} is represented by an elongated hexagon, as illustrated in \fig{af:phenotype}.

\glyphLabel A \glyph{phenotype} is identified by a label placed in an unbordered box containing a string of characters.  The characters can be distributed on several lines to improve readability, although this is not mandatory.  The label box must be attached to the centre of the \glyph{phenotype} container.  The label may spill outside of the container.

\glyphAux None.

\end{glyphDescription}

\begin{figure}[H]
  \centering
  \includegraphics[scale = 1]{images/build/phenotype.pdf}
  \caption{The \AF glyph for \glyph{phenotype}.}
  \label{fig:af:phenotype}
\end{figure}
