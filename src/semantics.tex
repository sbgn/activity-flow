\section{Glyph specific rules}

\subsection{Activity node}

 \begin{enumerate}
    \item Each \glyph{biological activity} must have at most one \glyph{unit of information}.
    \item \glyph{Phenotype} must not be the origin of an influence arc.
 \end{enumerate}

\subsection{Activity node and Compartment}

Each AN can only appear once in a particular compartment.  An AN can only \emph{belong} to one compartment. However, an AN can be \emph{drawn} over more than one compartment. In such cases the decision on which is the owning compartment is deferred to the drawing tool or the author. 

The layout of compartments in an SBGN diagram does not imply anything about the topology of compartments in the cell. Compartments must be bounded and may overlap. However, adjacency and the nesting of compartments do not imply that these compartments are next to each other physically or that one compartment contains the other.

\subsection{Influence}

It is implied, but not defined explicitly that an activity has a rate at which the input AN affects the output AN. 

\begin{enumerate}
\item The effect of a positive influence is to increase the basal activity.
\item The effect of a negative influence is to decrease the basal activity.
\item The effect of an unknown influence on the basal activity is unknown.
\item An activity can be targeted by at most one necessary stimulation. Combinations of more than one necessary stimulation must be explicitly expressed using the Boolean  AND or OR  operators.
\end{enumerate}

\subsection{Submaps}

Submaps are a visual device that allow a map to be split into several views. They remain, however, part of the main map and share its namespace. As a test of validity it should be possible to reintroduce a submap into the main map by eliminating the \glyph{submap terminals} and merging
the equivalent nodes in both maps.

\subsubsection{Rules for mapping to submaps}

An AN in the main map can be mapped to one in the submap using a \emph{tag} in the submap and \glyph{submap terminals} (see \sect{submap}) in the main map. For a mapping between map and submap to exists the following must be true:

\begin{enumerate}
\item The identifiers in the \emph{tags} and \glyph{submap terminals} must be identical.
\item The ANs must be identical.
\end{enumerate}

\subsubsection{Requirement to define a mapping}

If a map and submap both contain the same AN, then a mapping between them must be defined as above.